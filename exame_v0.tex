\begin{enumerate}
	% question 1
	\item Who is responsible for managing the Product Backlog?
	\begin{todolist}
		\item The Product Owner
		\item The Scrum Master
		\item The Development Team
		\item The Key Stakeholders
	\end{todolist}

	% question 2
	\item It is a good practice to have at least two Product Owners on big projects.
	\begin{todolist}
		\item False
		\item True
	\end{todolist}

	% question 3
	\item What happens during the Sprint? Select three answers.
	\begin{todolist}
		\item No changes are made that would endanger the Sprint Goal
		\item Quality goals do not decrease
		\item Scope may be clarified and re-negotiated between the Product Owner and Development Team as more is learned
		\item Sprint scope is defined at the Sprint Planning and cannot be changed
		\item The Sprint Goal is changed frequently to reflect the status of the remaining work
	\end{todolist}

	% question 4
	\item Who has the authority to cancel the Sprint?
	\begin{todolist}
		\item The Scrum Master
		\item The Product Owner
		\item The Development Team
		\item The Key Stakeholders
		\item The Product Owner and the Scrum Master
	\end{todolist}

	% question 5
	\item What provides guidance to the Development Team on why it is building the Increment?
	\begin{todolist}
		\item The Sprint Goal
		\item The Scrum Master
		\item The Product Owner
		\item The Sprint Backlog
	\end{todolist}

	\newpage

	% question 6
	\item Who participates in the Sprint Review? Select all applicable variants.
	\begin{todolist}
		\item The Scrum Master
		\item The Product Owner
		\item The Development Team
		\item The Key Stakeholders
		\item The Organization CEO
	\end{todolist}

	% question 7
	\item Who is responsible for monitoring progress toward high-level goals?
	\begin{todolist}
		\item The Product Owner
		\item The Product Owner and The Development Team
		\item The Scrum Master and The Development Team
		\item The Scrum Master
		\item The Development Team
		\item The Scrum Team
	\end{todolist}

	% question 8
	\item What are the Scrum Artifacts? Select all applicable items.
	\begin{todolist}
		\item Product Backlog
		\item Sprint Backlog
		\item Increment
		\item The list of removed impediments
		\item The Sprint Goal
	\end{todolist}

	% question 9
	\item What could be a source of requirements for any changes to be made to the product?
	\begin{todolist}
		\item The Product Backlog
		\item The CEO of the Organization
		\item The Key Stakeholders
	\end{todolist}

	% question 10
	\item Who is responsible for the Product Backlog?
	\begin{todolist}
		\item The Product Owner
		\item The Product Owner and The Scrum Master
		\item The Scrum Master
		\item The Product Owner and The Development Team
		\item The Scrum Master and The Development Team
		\item The Development Team
	\end{todolist}

	\newpage

	% question 11
	\item What are Product Backlog features? Select three.
	\begin{todolist}
		\item It is never complete
		\item It is dynamic
		\item As long as a product exists, its Product Backlog also exists
		\item When the final version of a product is rolled out, its Product Backlog is dismissed
		\item A Product Backlog could be closed when it contains no items to include into the next Sprint
	\end{todolist}

	% question 12
	\item All Development Teams working on the same Product should use the same Product Backlog.
	\begin{todolist}
		\item True
		\item False
	\end{todolist}

	% question 13
	\item Who is responsible for all estimates in the Product Backlog?
	\begin{todolist}
		\item The Development Team
		\item The Product Owner
		\item The Scrum Team
		\item The Scrum Master
		\item The Product owner and the Development Team
		\item The Scrum Master and the Development Team
		\item The Product owner and the Scrum Master
	\end{todolist}

	% question 14
	\item What is the Sprint Backlog?
	\begin{todolist}
		\item The Product Backlog items selected for this Sprint plus the plan for delivering them
		\item The Product Backlog items selected for this Sprint
		\item The Product Backlog items selected for this Sprint plus the Team Backlog items
		\item The plan for delivering Product Backlog items
	\end{todolist}

	\newpage
	
	% question 15
	\item Who is responsible for tracking the total work remaining in the Sprint Backlog to project the likelihood of achieving the Sprint Goal?
	\begin{todolist}
		\item The Development Team
		\item The Product Owner
		\item The Scrum Team
		\item The Scrum Master
		\item The Product Owner and the Development Team
	\end{todolist}

	% question 16
	\item Who is allowed to change the Sprint Backlog during the Sprint?
	\begin{todolist}
		\item The Development Team
		\item The Product Owner
		\item The Scrum Team
		\item The Scrum Master
		\item The Development Team and the Product Owner
	\end{todolist}

	% question 17
	\item What is the Increment?
	\begin{todolist}
		\item The sum of all the Product Backlog items completed during the Sprint
		\item The sum of all the Product Backlog items completed during the Sprint and the value of the increments of all previous Sprints
		\item All "Done" items in the Sprint Backlog
		\item All items in the Sprint Backlog that could be released regardless of whether the Product Owner decides to actually do it
	\end{todolist}

	% question 18
	\item Who is responsible for creation of the Definition of  Done ?
	\begin{todolist}
		\item The Development Team
		\item The Scrum Team
		\item The Product Owner
		\item The Scrum Master
	\end{todolist}

	% question 19
	\item Who is allowed to participate in the Daily Scrum?
	\begin{todolist}
		\item The Development Team
		\item The Scrum Master
		\item The Product Owner
		\item The Key Stakeholders
	\end{todolist}

	\newpage

	% question 20
	\item What does Burn-down Chart show?
	\begin{todolist}
		\item How much work remains till the end of the Sprint
		\item The evolution of the amount of uncertainty during a project
		\item Dependencies, start times and stop times for project tasks
		\item Hierarchy of tasks that comprise a project
	\end{todolist}

	% question 21
	\item What is the order of items in the Product Backlog?
	\begin{todolist}
		\item Alphabetical
		\item Less valuable and most unclear items at the bottom
		\item The recently added items at the top
		\item The less clear items at the top
	\end{todolist}

	% question 22
	\item All the Scrum Teams working on the same product should have the same Sprint length.
	\begin{todolist}
		\item False
		\item True
	\end{todolist}

	% question 23
	\item How does the Scrum Master help the Product Owner? Select the three most appropriate answers.
	\begin{todolist}
		\item Facilitating Scrum events as requested or needed
		\item Finding techniques for effective Product Backlog management
		\item Understanding product planning in an empirical environment
		\item Introducing cutting edge development practices
		\item Leading and coaching the organization in its Scrum adoption
	\end{todolist}

	% question 24
	\item What does Cone of Uncertainty show?
	\begin{todolist}
		\item Hierarchy of tasks that comprise a project
		\item How much work remains till the end of the Sprint
		\item How much is known about the Product over time
		\item Dependencies, start times and stop times for project tasks
	\end{todolist}

	% question 25
	\item If an item in the Sprint Backlog cannot be finished by the end of the Sprint (it turned out there is a lot more work to do than was estimated), the Sprint is cancelled.
	\begin{todolist}
		\item False
		\item True
	\end{todolist}

	% question 26
	\item How does Definition of  Done  help the Scrum Team? Select three most applicable items.
	\begin{todolist}
		\item DoD is used to assess when work is complete on the product Increment
		\item Guides the Development Team in knowing how many Product Backlog items it can select during a Sprint Planning
		\item DoD ensures artifact transparency
		\item DoD helps in inspection and adaptation
		\item DoD helps to calculate velocity of the Scrum Team
	\end{todolist}

	% question 27
	\item What part of the capacity of the Development Team does Product Backlog refinement usually consume?
	\begin{todolist}
		\item Not more than 10%
		\item Not more than 20%
		\item Not more than 5%
		\item The Development Team is not authorized for Product Backlog refinement
	\end{todolist}

	% question 28
	\item Select the two meetings in which people outside the Scrum Team are allowed to participate.
	\begin{todolist}
		\item The Sprint Planning
		\item The Sprint Review
		\item The Sprint Retrospective
		\item The Daily Scrum
	\end{todolist}

	% question 29
	\item What are the three most applicable characteristics of the Product Owner?
	\begin{todolist}
		\item Product Value Maximizer
		\item Lead Facilitator of Key Stakeholder Involvement
		\item Product Marketplace Expert
		\item Lead Scrum evangelist in the Organization
		\item Facilitator of Scrum events
	\end{todolist}

	% question 30
	\item The Sprint Backlog is created at the Sprint Planning. It is prohibited to add new work into the Sprint Backlog later by the Development Team.
	\begin{todolist}
		\item False
		\item True
	\end{todolist}

	% question 31
	\item In which meetings the Key Stakeholders are allowed to participate?
	\begin{todolist}
		\item The Sprint Review
		\item The Sprint Retrospective
		\item The Sprint Planning
		\item The Daily Scrum
	\end{todolist}

	% question 32
	\item Who is allowed to make changes in the Product Backlog? Select two options.
	\begin{todolist}
		\item The Product Owner
		\item The Development Team, but with permission of the Product Owner
		\item The Key Stakeholders
		\item Anyone
		\item The Scrum Master
	\end{todolist}

	% question 33
	\item Who is responsible for crafting the Sprint Goal at the Sprint Planning?
	\begin{todolist}
		\item The Product Owner
		\item The Scrum Team
		\item The Scrum Master
		\item The Development Team
		\item The Key Stakeholders
	\end{todolist}

	% question 34
	\item Who participates in the Sprint Planning? Select three.
	\begin{todolist}
		\item The Product Owner
		\item The Scrum Master
		\item The Development Team
		\item The Key Stakeholders
		\item The Team Manager
	\end{todolist}

	\newpage

	% question 35
	\item What happens when a Sprint is cancelled? Select three.
	\begin{todolist}
		\item Any completed and  Done  Product Backlog items are reviewed
		\item If part of the work is potentially releasable, the Product Owner typically accepts it
		\item All incomplete Product Backlog Items are re-estimated and put back on the Product Backlog
		\item Several top Product Backlog Items are taken into the Sprint Backlog to replace the obsolete items
		\item At the Sprint Retrospective the Scrum Master determines who from the Development Team is responsible for cancelling the Sprint
	\end{todolist}

	% question 36
	\item Could the Product Owner and the Scrum Master be a part of the Development Team?
	\begin{todolist}
		\item Yes
		\item No
	\end{todolist}

	% question 37
	\item What does Product Backlog management include? Select three most applicable items.
	\begin{todolist}
		\item Optimizing the value of the work the Development Team performs
		\item Ensuring that the Product Backlog is visible, transparent, and clear to all, and shows what the Scrum Team will work on next
		\item Ordering the items in the Product Backlog to best achieve goals and missions
		\item Moving Product Backlog items into the Sprint Backlog
		\item Presenting Product Backlog items to the Key Stakeholders
	\end{todolist}

	% question 38
	\item The Scrum Team consists of
	\begin{todolist}
		\item The Scrum Master
		\item The Product Owner
		\item The Development Team
		\item The Key Stakeholders
	\end{todolist}

	\newpage

	% question 39
	\item Who is allowed to tell the Development Team to work from a set of requirements?
	\begin{todolist}
		\item The Product Owner
		\item The Scrum Master
		\item The Key Stakeholders
		\item Upper Management
		\item The Product Owner and the Scrum Master
	\end{todolist}

	% question 40
	\item The Development Team should be able to explain to the Product Owner and Scrum Master how it intends to work as a self-organizing team to accomplish the Sprint Goal and create the anticipated Increment.
	\begin{todolist}
		\item False
		\item True
	\end{todolist}

	% question 41
	\item Product Backlog Refinement  
Select the three most applicable sentence endings.
	\begin{todolist}
		\item Is the act of adding detail, estimates, and order to Product Backlog items
		\item Is an ongoing process
		\item Usually happens 2-4 times in dependency of the Sprint length
		\item Is time-boxed to a maximum of 4 hours
		\item Answers the question: how will the work needed to deliver the Increment be achieved
		\item Usually takes no more than 10\% of the capacity of the Development Team
	\end{todolist}

	% question 42
	\item Select the two focus areas that are not considered in executing Value Driven Development by the Product Owner.
	\begin{todolist}
		\item Product Value Maximizer
		\item Product Visionary
		\item Product Marketplace Expert
		\item Product Release Decision Maker
		\item Lead Facilitator of Key Stakeholder Involvement
		\item Coach of the Development Team in self-organization and cross-functionality
		\item Remover of impediments to the Development Team s progress
	\end{todolist}

	% question 43
	\item Which KVA categories should the Product Owner consider to measure and track the creation and delivery of value to the market place (select three)?
	\begin{todolist}
		\item Current Value
		\item Time-to-Market
		\item Ability to Innovate
		\item Risk Reduction
		\item Employee Satisfaction
		\item Capability Building
	\end{todolist}

	% question 44
	\item Who is the chief product visionary?
	\begin{todolist}
		\item The Product Owner
		\item The Scrum Master
		\item The Chief Executive Officer (CEO)
		\item The Chief Marketing Officer (CMO)
	\end{todolist}

	% question 45
	\item How can the Product Owner bring his product vision to life (select 3)?
	\begin{todolist}
		\item Utilizing the underlying empirical product planning features of Scrum
		\item Via the Product Backlog and iterating towards that vision every Sprint
		\item Articulating the product vision to the Scrum Team and the Key Stakeholders early and often
		\item Asking for an approval of the Upper Management
		\item Making the Scrum Master bring the vision to the Scrum Team and the Key Stakeholders
	\end{todolist}

	% question 46
	\item How frequently the Product Owner should communicate and re-iterate his product vision to the Scrum Team and the Key Stakeholders?
	\begin{todolist}
		\item Early and often
		\item Once at the first Sprint Planning
		\item Every Daily Scrum
		\item Every Sprint Retrospective
	\end{todolist}

	\newpage

	% question 47
	\item The Product Owner should be expertly aware of the marketplace for the product.
	\begin{todolist}
		\item True
		\item False
		\item It depends
	\end{todolist}

	% question 48
	\item Who should do the legwork of gathering the marketplace data for the Product Owner?
	\begin{todolist}
		\item It does not matter who does the legwork
		\item The Product Owner
		\item The Scrum Team
		\item The Scrum Team and the Key Stakeholders
	\end{todolist}

	% question 49
	\item How does the Product Owner communicate his marketplace knowledge to the Scrum Team (select three)?
	\begin{todolist}
		\item Daily ad hoc interactions
		\item Product Backlog Refinement
		\item Sprint Reviews
		\item Daily Scrums
		\item Sprint Retrospectives
	\end{todolist}

	% question 50
	\item Once the Product Owner gained his Product Vision and defined the tactics of bringing this vision to life, it is a bad idea to change them before the next Product Release.
	\begin{todolist}
		\item False
		\item True
	\end{todolist}

	% question 51
	\item Who decides whether to release the latest increment of the product?
	\begin{todolist}
		\item The Product Owner
		\item The Scrum Master
		\item The Scrum Team
		\item The Development Team
		\item The Product Owner and The Scrum Master
	\end{todolist}

	\newpage

	% question 52
	\item How frequently product releases should occur?
	\begin{todolist}
		\item Frequently enough to eliminate the risk that the product s value will get out of line with the marketplace
		\item Every Sprint
		\item By the end of Product development
		\item Every 3 months
		\item At least every 6 months
	\end{todolist}

	% question 53
	\item What factors should be considered by the Product Owner in the release decision (select four)?
	\begin{todolist}
		\item The risk that the product s value can get out of line with the marketplace
		\item Can customers actually absorb the new release?
		\item The costs and benefits of the upgrade
		\item The customers that will be constrained by the new release
		\item The amount of work remaining toward the Sprint Goal
		\item Approval of the Key Stakeholders
		\item Does the Increment meet the Definition of  Done ?
	\end{todolist}

	% question 54
	\item Who identifies the Key Stakeholders for the Product?
	\begin{todolist}
		\item The Product Owner
		\item The Scrum Master
		\item The Development Team
		\item The Scrum Team
		\item The Upper Management
	\end{todolist}

	% question 55
	\item Who are the typical Key Stakeholders (select three)?
	\begin{todolist}
		\item The human people who actually use the product under development
		\item The people responsible for paying to use the product
		\item The people responsible for making the funding decisions for the product development effort
		\item The people responsible for product development
		\item The people responsible for product marketing
	\end{todolist}

	\newpage

	% question 56
	\item When is the Scrum Team allowed to interact with the Key Stakeholders (select the most applicable option)?
	\begin{todolist}
		\item The Sprint Review
		\item Any time where it s valuable to have the Stakeholder input
		\item The Daily Scrum
		\item The Sprint Retrospective
	\end{todolist}

	% question 57
	\item If multiple Stakeholders have varied interests in the product and different viewpoints what is the best strategy for the Product Owner?
	\begin{todolist}
		\item Do an intelligent balancing of interests and try to maximize the value of the Product as a whole
		\item Listen to the people that fund the product s development because they always have the last word
		\item Stick to the viewpoint promising the fastest time-to-market
		\item Calculate ROI (Return Of Investments) for every viewpoint and select the maximal one
	\end{todolist}

	% question 58
	\item When a product grows, it is quite possible that the PO will get help from other Product Managers and others in the organization who interact regarding the customer facing activities and knowledge of the product marketplace. Is it a good idea for the PO to proxy or outsource some of their PO Scrum Team duties to these people (for example, Scrum Team facing duties)?
	\begin{todolist}
		\item No
		\item Yes
	\end{todolist}

	% question 59
	\item When something about Scrum frustrates the Product Owner, the PO can delegate some responsibilities to the Scrum Master.
	\begin{todolist}
		\item False
		\item True
	\end{todolist}

	\newpage

	% question 60
	\item What does  the word  development  mean in the context of Scrum? Select the best option.
	\begin{todolist}
		\item Software and hardware development
		\item Product development, its releasing and sustaining
		\item Development of an operational environment for the Product
		\item Research and identifying of viable markets, technologies, and Product capabilities
		\item Complex work that can include all the suggested options and even more
	\end{todolist}

	% question 61
	\item Where Scrum can be used? Check all the applicable items.
	\begin{todolist}
		\item Research and identifying of viable markets, technologies, and product capabilities
		\item Development and sustaining of Cloud and other operational environments
		\item Development of software and hardware
		\item Development of products and enhancements
		\item Managing the operation of an organization
		\item Development of almost everything we use in our daily lives as individuals and societies
	\end{todolist}

	% question 62
	\item What is the essence of Scrum? Select the most appropriate option.
	\begin{todolist}
		\item A small team of people that is highly flexible and adaptive
		\item The Scrum Guide
		\item The Development Team
		\item The Scrum Master and the Product Owner
	\end{todolist}

	% question 63
	\item Select the five Scrum Values.
	\begin{todolist}
		\item Commitment
		\item Courage
		\item Focus
		\item Openness
		\item Respect
		\item Self-organization
		\item Effectiveness
		\item Agility
	\end{todolist}

	% question 64
	\item Who is responsible for promoting and supporting Scrum? Select the best choice.
	\begin{todolist}
		\item The Scrum Master
		\item The Product Owner
		\item The Scrum Master and the Product Owner
		\item The Scrum Team
		\item The Development Team
	\end{todolist}

	% question 65
	\item Imagine the following situation. At the Sprint Retrospective meeting the Scrum Team identified some improvements that can be done. What should the Scrum Team do? Select the best option.
	\begin{todolist}
		\item Make sure the Sprint Backlog for the next Sprint includes at least one high priority process improvement.
		\item Make sure the Sprint Backlog for the next Sprint includes all the improvements.
		\item Assign responsible team members for every improvement. Check the progress at the next Retrospective.
		\item Assign a responsible team member for at least one improvement. Check the progress at the next Retrospective.
	\end{todolist}

	% question 66
	\item Who has the  last say  on the order of items in the Product Backlog?
	\begin{todolist}
		\item The Product Owner
		\item The Scrum Master
		\item The Development Team
		\item The Product Owner and The Scrum Master
	\end{todolist}

	% question 67
	\item What technique should be used to represent Product Backlog Items?
	\begin{todolist}
		\item Any technique, even a mix of several techniques
		\item User Stories
		\item Use Cases
		\item Scenarios
		\item Acceptance Tests
	\end{todolist}

	% question 68
	\item Every Product Backlog Item should be created by the Product Owner personally and only then the Development Team can add details to it at the PO s discretion.
	\begin{todolist}
		\item False
		\item True
	\end{todolist}

	% question 69
	\item Product Backlog Refinement practice focuses on Items for upcoming Sprints, not the current Sprint in progress. True or false?
	\begin{todolist}
		\item True
		\item False
	\end{todolist}

	% question 70
	\item What are the characteristics of a Product Backlog Item that is  Ready  for selection in a Sprint Planning? Select three.
	\begin{todolist}
		\item Can be "Done" within one Sprint
		\item Somewhere at the top of the Product Backlog
		\item Well refined
		\item Somewhere at the bottom of the Product Backlog
		\item Can be implemented within one Sprint and tested in the next Sprint
		\item Has less detail
	\end{todolist}

	% question 71
	\item Who is the leader in terms of getting feedback from the Key Stakeholders in the Sprint Review?
	\begin{todolist}
		\item The Product Owner
		\item The Development Team
		\item The Scrum Master
		\item The Scrum Team
	\end{todolist}

	% question 72
	\item The Sprint Review is just a demo of the Product Backlog items completed during a Sprint. Do you agree?
	\begin{todolist}
		\item No, the Sprint Review contains much more activities
		\item No, the demo also should include the Items completed in the previous Sprints that were not demonstrated for some reason
		\item Yes. There is no much difference.
	\end{todolist}

	% question 73
	\item Select the three best options to finish the sentence below.
Technical debt  
	\begin{todolist}
		\item is a real risk which can genuinely be incurred
		\item compromises long-term quality of the Product
		\item reflects some extra development work
		\item belongs entirely to the Development Team. No one else should know about it.
		\item is a lack of technical supplies
	\end{todolist}

	% question 74
	\item The Scrum Master should not allow the Product Owner to attend the Sprint Planning if the PO is not ready with a Sprint Goal. Is this true or false?
	\begin{todolist}
		\item False
		\item True
	\end{todolist}

	% question 75
	\item What two attributes are optional for a Product Backlog Item?
	\begin{todolist}
		\item Description
		\item Order
		\item Estimate
		\item Value
		\item Test descriptions that will prove PB Item completeness when "Done"
		\item Dependencies
	\end{todolist}

	% question 76
	\item How long does the Product Backlog exists?
	\begin{todolist}
		\item While the Product exists
		\item Not more than 5 years
		\item Till the final Product Release
		\item While at least one Development Team is working on it
	\end{todolist}

	% question 77
	\item A Development Team is waiting for a specific software component that they need to integrate and use.
The component should be ready in a month.
The Backlog Items with highest priorities depend on this specific component.
What should the Product Owner do?
	\begin{todolist}
		\item Make sure the dependency is visible in the Product Backlog and the Development Team has enough independent Items for the next Sprint.
		\item Nothing. The Product Backlog already has the most valuable items at the top. The Development Team cannot proceed further until the dependency is resolved.
		\item Remove the dependent Items from the Product Backlog and put them in a special wait list. When the dependency is resolved, the Items should be returned back.
		\item Transfer the dependent Items to the Integration Team
	\end{todolist}

	\newpage

	% question 78
	\item The Product Owner wants to apply some non-functional requirements to the Product. What is the best way to proceed?
	\begin{todolist}
		\item Add the non-functional requirements to the DoD and check every Increment against these criteria
		\item Create a new Item for every requirement in the Product Backlog
		\item Non-functional requirements can't be handled within the bounds of Scrum
		\item Find a way to convert non-functional requirements into Product features and act accordingly
	\end{todolist}

	% question 79
	\item What are the time-boxes for the Sprint Review and the Sprint Retrospective?
	\begin{todolist}
		\item 4 and 3 hours respectively
		\item 3 and 4 hours respectively
		\item 3 hour time-box for each
		\item 4 hour time-box for each
	\end{todolist}

	% question 80
	\item How long should the Sprint Planning be?
	\begin{todolist}
		\item Not more than 8 hours
		\item Not more than 4 hours
		\item Not more than 10% of the capacity of the Development Team
		\item Until all the Items in the Sprint Backlog are decomposed to units of one day or less
	\end{todolist}

\end{enumerate}